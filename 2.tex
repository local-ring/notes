Fisrt of all, I would like to say that I don't think I understand the lecture very well and this notes is therefore not ``well-done''. So it would be great if you want to improve it (this is the only reason why I \LaTeX{} it). I put the \TeX{} source code on \texttt{GitHub} so that you can modify and elaborate them:\begin{center}
 \url{https://github.com/local-ring/notes/blob/master/2.tex}   
\end{center}

Furthermore, thanks to Prof. Weber's pictures, as well as his Mathematica package and documentation.\\
\par


We are going to study the Riemann surfaces associated to the following algebraic functions:
\begin{itemize}
    \item Fermat : $y^4=x^4-1$
    \item Klein:\ \  \ \  $y^7=x(x-1)^2(x-2)^4$
    \item ?:\ \ \   \quad \quad   $y^3=x^6-1$
\end{itemize}

At first, we consider the Klein's Riemann surface $X$
$$Y^7=x(x-1)^2(x-2)^4$$

It is easy to see there are three branched points: $0$, $1$, and $2$. Near them, the map $X\rightarrow \PP^{1}$ should respectively looks like 

\[\begin{aligned}
    Y & \mapsto \zeta = e^{\frac{2\pi i}{7}}\\
    Y & \mapsto \zeta^{2} = e^{2\frac{2\pi i}{7}}\\
    Y & \mapsto \zeta^{4} = e^{4\frac{2\pi i}{7}}
\end{aligned}\]

Since $7$ is a prime number, which is nice, the map is therefore of branched index $6$ at each branched point. Applying Riemann-Hurwitz formula, the genus of Klein's Riemann surface should be 
\[\begin{aligned}
g^{\mathrm{Klein}}&=7(0-1)+1+\frac{1}{2}(6+6+6)\\
&=-6+9\\
&=3
\end{aligned}\]
where the seven and the zero in the first row are degree of the holomorphic map and the genus of $\PP^1$ respectively. \par

According to \cite{MR1722412}, the automorphism group of Klein's surface is 
\[\Aut(\mathrm{Klein})=\left\{\frac{az+b}{cz+d}~~\bigg|\,\ \begin{pmatrix}a&b\\c&d\end{pmatrix}\in \mathrm{PSL}_2(\Z) \text{~~and ~~}\begin{pmatrix}a&b\\c&d\end{pmatrix}\equiv \begin{pmatrix}1&0\\0&1\end{pmatrix} \pmod{7}\right\}\]

Also, the order of the above group is $24\cdot 7=168$. 
\\ 
\par


To find the automorphism of the Klein's Riemann surface, we can start with drawing its hyperbolic picture. Consider the upper plane, using Riemann mapping theorem we can map $0$, $1$ and $2$ to three points on the unit circle (we obtain a zero triangle, i.e. every angle is $0$). Then, extend the mapping by reflection and do some necessary shrinking, we obtain a equilateral hyperbolic triangle with the angle $\frac{\pi}{7}$. After reflections, we can get the hyperbolic structure of Klein's Riemann surface (See the following figures. They are produced by Prof. Weber's Mathematica package).
\newpage
\begin{figure}[h!]
\centering
\includegraphics[width=6cm]{tess2.pdf}
\includegraphics[width=6cm]{tess1.pdf}
\caption{Tessallations}
\end{figure}

\par



Alternatively, we can consider the Euclidean triangle with angles $\frac{\pi}{7}$, $\frac{2\pi}{7}$ and $\frac{4\pi}{7}$, based on the observation that sum all angles up and we get exactly $\pi$. Beginning with the Euclidean triangle, do reflections and some Jigsaw puzzle, we have the planar picture of Klein's surface, consisting of 14 pieces of the triangle.  

Here is the advantage of dealing with the Euclidean triangle (See \cite{MR1722412}):
\begin{quote}
The certain quotients of Klein’s surface $X$ are rhombic toriand we would like to know more about them. While we don’t have any arguments using hyperbolic geometry to obtain this information, there is a surprisingly simple way using flat geometry. The idea is as follows: Suppose we have a holomorphic map from $X$ to some torus. Its exterior derivative will be a well defined holomorphic $1$-form on $X$ with the special property that all its periods lie in a lattice in $\C$. Vice versa, the integral of such a $1$-form will define a map to a torus whose lattice is spanned by the periods of the $1$-form. There are two
problems with this method: 

It is rarely the case that one can write down holomorphic 1-forms for a Riemann surface. An exception are the hyperelliptic surfaces in their normal form $y^2 = P(x)$ where one can multiply the meromorphic form $\rd x$ by rational functions in $x$ and $y$ to obtain a basis of holomorphic forms. But even if one can find holomorphic $1$-forms then it is most unlikely that one can integrate them to compute their periods.

Flat geometry helps to overcome both problems simultaneously: Any holomorphic $1$-form $\omega$ determines a flat metric on the surface that is singular in the zeroes of $\omega$ and that has trivial linear holonomy (parallel translation around any closed curve is the identity). This flat metric can be given by taking $|\omega|$ as its line element. Another way to describe it is as follows: Integrate $\omega$ to obtain a locally defined map from the surface to $\C$. Use this map to pull back the metric from $\C$ to the surface. A neighborhood around a zero of order $k$ of $\omega$ is isometric to a euclidean cone with cone angle $2\pi(k + 1)$, as can be seen in a local coordinate.
    
\end{quote}

By the above statement, if we do the Jigsaw puzzle to each vertex, then we have the vertex of $\frac{2\pi}{7}$ angle is a simple zero and the vertex of $\frac{4\pi}{7}$ angle is the zero of order $3$.

Similar with the hyperbolic procedure, we have a map from the (``punctured'') upper half plane to the triangle, which is a Schwarz-Christoffel mapping, given by,
\[f(z)=\int_{i}^{z} x^{\frac{1}{7}-1} (x-1)^{\frac{2}{7}-1} (x-2)^{\frac{4}{7}-1}\rd x\]
Then the holomorphic $1$-form on $X$ will be 
\[\begin{aligned}
\omega_1 &= x^{\frac{1}{7}-1} (x-1)^{\frac{2}{7}-1} (x-2)^{\frac{4}{7}-1}\rd x = \frac{Y}{x(x-1)(x-2)}\rd x \\
\omega_2 &= x^{\frac{2}{7}-1} (x-1)^{\frac{4}{7}-1} (x-2)^{\frac{1}{7}-1}\rd x \\
\omega_3 &= x^{\frac{4}{7}-1} (x-1)^{\frac{1}{7}-1} (x-2)^{\frac{2}{7}-1}\rd x \end{aligned}\]

Using Mathematica, we can compute the Wronskian determinant so that we can get Weierstrass points. Then we have the following table:
\begin{table}[h!]
    \centering
    \begin{tabular}{c|ccc}
        &  0&1&2\\ \hline 
       $\omega_1$  & 0&1 &3  \\
       $\omega_2$  & 1&3 &$\star$\\
       $\omega_3$  & 3&$\star$ &1
    \end{tabular}
\end{table}

which consists of the weight of Weierstrass points. The gaps at $0$, $1$ and $2$ are $1$, $2$ and $4$ respectively. Also, by the formula $W=(g-1)g(g+1)$, we have $X$ has $24$ Weierstrass points. \\ 
\par 
Now, we are going to study the Riemann surface associated to 
\[Y^{3}=x^6-1\]

We start with a $6$ points punctured plane and connect the points (then get the slits) in the following way:
\[\zeta_1 - \zeta_2 ,\quad \zeta_3 - \zeta_4,\quad \zeta_5 - 1 \]
By Riemann mapping theorem, we can map the punctured plane to a hexagon and do reflections. Then we obtain

\begin{figure}[h!]
\centering
\includegraphics[width=8cm]{hexa-annulus.pdf}
\end{figure}

Similarly, we have the genus of the Riemann surface is $4$ and we can calculate its Weierstrass points.

